\textbf{Abstract}
A new low emittance lattice storage ring is under construction at the ESRF.
In this new instrument, an upgraded end station for ID31 beamline must allow to position the samples along \emph{complex trajectories with a nanometer precision}.
In order to reach these requirements, samples have to be mounted on high precision stages, combining a capability of large stroke, spin motion, and active rejection of disturbances.
However, the precision is limited by thermal expansion and various imperfections
that are not actively compensated. Our approach is to add a Nano Active Stabilization System (NASS) which is composed of a 6DoF Stewart platform and a 6 DoF metrology system.
A 3D model of the end station updated with experimental data is developed.
As the mass of the samples may vary by up to two orders of magnitudes, robust control strategies are required to address such plant uncertainty.
The proposed control strategy are presented and applied on the developed model by conducting time domain simulations of tomography experiment in presence of instrumentation noise and system uncertainty.

%%% Local Variables: ***
%%% mode:latex ***
%%% TeX-master: "2018 - Student Day.tex"  ***
%%% End: ***